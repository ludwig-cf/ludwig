%%%%%%%%%%%%%%%%%%%%%%%%%%%%%%%%%%%%%%%%%%%%%%%%%%%%%%%%%%%%%%%%%%%%%%%%%%%%%%%
%
%  colloid.tex
%
%%%%%%%%%%%%%%%%%%%%%%%%%%%%%%%%%%%%%%%%%%%%%%%%%%%%%%%%%%%%%%%%%%%%%%%%%%%%%%%

\section{Colloid Examples}


\subsection{Initialisation from file}

We present a number of examples taken from the regression tests, the inputs
for which are found in
\begin{verbatim}
trunk/tests/regression
\end{verbatim}
For each test there is an input file, and an ASCII file containing the
details of the initial colloid state which is read at run time by means
of the key value pair
\begin{verbatim}
colloid_init    from_file
\end{verbatim}

A file of colloid information may be created with the utility
\begin{verbatim}
util/colloid_file.c
\end{verbatim}
As a minimum, this file must specify:
\begin{enumerate}
\item
A unique integer id for each particle;
\item
a radius and hydrodynamic radius $a_0$ and $a_h$ (use the same value
if unsure);
\item
an initial position ($x_{min} < x < x_{max}$) with $x_{min} = 0.5$
and $x_{max} = L_x + 0.5$ etc, where $L_x$ is the appropriate system
size for the problem at hand;
\item
the initial velocity etc may be safely initialised to zero.
\end{enumerate}
Note that is an inter-particle potential is to be specified at run time,
the initial position of the particles should not be so close that a
large force is experienced at time $t=0$. This can destabilise the
dynamics.

A single file of colloid information is produced in either ASCII or binary
format, which can be read in by specifying the appropriate
\begin{verbatim}
colloid_io_format_input   ASCII_SERIAL
\end{verbatim}
or \texttt{BINARY\_SERIAL} key value in both serial and parallel.


\subsubsection{Very short range potential}

See
\begin{verbatim}
tests/regression/test_spin_solid2_input
tests/regression/config.cds.init.001-001
tests/regression/test_spin_solid2_d3q19.ref*
\end{verbatim}

An example of a moderate volume fraction of particles is given for
a binary fluid undergoing spinodal decomposition. Here,
the capillary interaction between the neutrally wetting colloids
causes a strong effective attraction between particles. It is therefore
necessary to ensure the particles do not collide to the point of
overlapping at their hard-sphere radius. A counterbalancing short-range
soft-sphere potential is then defined in the input file:
\begin{verbatim}
soft_sphere_on 1
soft_sphere_epsilon 0.0004
soft_sphere_sigma 0.1
soft_sphere_nu 1.0
soft_sphere_cutoff 0.25
\end{verbatim}
where, following Eq.~\ref{eq_ss_shift}, we have
$v(r) \sim \epsilon (\sigma /r)^\nu$ ``cut-and-shifted'' so that
both potential and force smoothly match to zero at the cut-off
distance $r_c$ (here $0.25\Delta x$). The energy scale $\epsilon$
will need to be adjusted depending on the exact problem at hand
(here it will be related to the fluid-fluid interfacial tension).

Other relevant parameters here are
\begin{verbatim}
colloid_cell_min 8.0
lubrication_on 0
\end{verbatim}
which control, respectively, the minimum cell list width used to
help identify interactions, and the lubrication correction (here
switched off).


\subsubsection{Short-range potential}

See
\begin{verbatim}
tests/regression/test_yukawa_input
tests/regression/test_yukawa_cds.001-001
tests/regression/test_yukawa_d3q10.ref1
\end{verbatim}

This example involves a number of interacting particles in a simple
fluid including Brownian motion. The potential is Yukawa-like,
representative of a screened Coulomb interaction.

We have the parameters:
\begin{verbatim}
yukawa_on 1
yukawa_epsilon  1.330764285
yukawa_kappa 0.72463768115
yukawa_cutoff 16.0
\end{verbatim}
where the potential is $v(r) \sim \epsilon (-\kappa r) / r$. Again,
both potential and force are smoothly matched to zero at the
cut-off distance $r_c$ (here $r_c = 16\Delta x$). The relatively
long cut-off distance means that the minimum cell list cell width
must be correspondingly large (16 lattice units) which limits the
minimum parallel domain size.

This example also includes Brownian motion by means of fluctuating
hydrodynamics. The relevant parameters in the input are:
\begin{verbatim}
isothermal_fluctuations on 
temperature 0.0002133333
\end{verbatim}
Note that the \texttt{temperature} parameter here specifies
$k_BT = \langle c_x^2\rangle = \langle c_y^2 \rangle = \langle c_z^2 \rangle$
in three dimensions (at equilibrium);
this is reported in the output of the code at run time.

\subsubsection{Long range forces}

An Ewald sum is available for magnetic dipoles.
