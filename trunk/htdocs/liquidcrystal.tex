\section{Liquid Crystal}

\subsection{User input}

\inputkey{free\_energy lc\_blue\_phase}

Here, we have a tensor order parameter $Q_{\alpha\beta}$.
The free energy density is
\begin{eqnarray}
f = {\textstyle\frac{1}{2}}A_0(1 - \gamma/3)Q^2_{\alpha\beta}
  - {\textstyle\frac{1}{3}}A_0 \gamma
Q_{\alpha\beta}Q_{\beta\delta}Q_{\delta\alpha}
 + {\textstyle\frac{1}{4}}A_0 \gamma (Q^2_{\alpha\beta})^2
\nonumber
\\
+ {\textstyle\frac{1}{2}} \Big(
\kappa_0 (\epsilon_{\alpha\delta\sigma} \partial_\delta Q_{\sigma\beta} +
2q_0 Q_{\alpha\beta})^2 + \kappa_1(\partial_\alpha Q_{\alpha\beta})^2 \Big)
\end{eqnarray}

The corresponding input parameters are:

\inputkey{lc\_a0} The bulk free energy parameter $A_0$

\inputkey{lc\_gamma} The bulk free energy parameter $\gamma$

\inputkey{lc\_q0} The pitch wavenumber $q_0 = 2\pi / p$, where p
is the pitch.

\inputkey{lc\_kappa0} Elastic constant $\kappa_0$ in distortion free energy

\inputkey{lc\_kappa1} Elastic constant $\kappa_1$ in distortion free energy

Note that the code currently enforces the `one elastic constant'
approximation ($\kappa_0 = \kappa_1$), so both these values must
be equal in the input. This constraint may be relaxed in the future.


\inputkey{lc\_Gamma} The rotational diffusion constant appearing in
the Beris Edwards equation (note \texttt{lc\_gamma} and \texttt{lc\_Gamma}
are different).

\inputkey{lc\_xi} Aspect ratio for rod lie molecules appearing in the
stress tensor.

\inputkey{lc\_active\_zeta} Apolar liquid crystal activity constant.

\subsubsection{Initial LC order parameter $Q_{\alpha\beta}$}

\inputkey{lc\_q\_initialisation}

There are a number of useful choices here. The key value
\texttt{twist} initialises a simple cholesteric with the
helical axis in the $z-$direction, whereas the values 
\texttt{cholesteric\_x\_y\_z} provide similar initialisations 
with twist along the other axes.
The values \texttt{o8m} and \texttt{o2} give equilibrim BPI and BPII.
Another, cubic body-centered BP is obtained by the value \texttt{o5}.
Hexagonal BPs are initialised with \texttt{h2d, h3da} and \texttt{h3db}.\\
BPIII emerges from a start configuration of randomly positioned 
and oriented DTCs embedded into an isotropic or cholesteric environment.
This is initialised with the key value \texttt{bp3}, whereas the number
$N$ and radius $R$ of the DTCs and the type of the environment 
$ENV=\{0,1\}$, (0 = isotropic, 1 = cholesteric) is specified 
with the key value \texttt{lc\_init\_bp3} $N\_R\_ENV$. 
Note that calculations can become surprisingly unstable if the 
start configuration consits of too many and too large DTC-regions 
with respect to system size and decomposition.\\
The appropriate choices for the other parameters must be made correspondingly.

An additional \texttt{lc\_q\_init\_amplitude} value sets the
scalar order parameter  `amplitude' in each case. Typical
values are given in the input file. Values which are too large can
equally lead to instability.

\subsubsection{The redshift}

\inputkey{lc\_init\_redshift}

Sets the initial value of the redshift parameter. For example,
for the \texttt{o8m} initialisation, the redshift can be set to 0.83,
and for the \texttt{o2} initialisation, the redshift can be set to 0.91.
The default redshift is unity.

\inputkey{lc\_redshift\_update [0|1]}

Switches on (default is off) the dynamic computation of the redshift
from the order parameter gradient terms in the free energy. The
caluclation is performed at every time step, and override any
initial value specified in the above. It is also robust to restarts.

