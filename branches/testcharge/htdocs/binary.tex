\section{Binary Fluids}

\subsection{The Cahn-Hilliard Equation}

\subsection{The Choice of Free Energy}

\subsection{A Second Distribution Function}

\subsection{The Collision}

\subsection{Bounce-Back on Links}

\subsection{Upwind Advection Schemes}

The solution to the Cahn-Hilliard equation for the order parameter
\begin{equation}
\partial_t \phi + \partial_\alpha (u_\alpha \phi + M\partial_\alpha \mu) = 0
\label{eq:ch}
\end{equation}
assumes that the valocity field $u_\alpha$ is known, along with the
order parameter mobility $M$. Adopting a divergence form ensures that
any finite difference scheme conserves the total order parameter in
the system.

Considering just the advective part of \ref{eq:ch}, a finite difference
approach boils down to finding an interpolation of $\phi(\mathbf{r})$ to
the faces of a control volume surrounding the lattice site at $\mathbf{r}$.
So, in one dimension we have
\begin{equation}
\phi_i^{n+1} = \phi_i^n + \frac{\Delta t}{\Delta x} (u_w \phi_w - u_e \phi_e)
\label{eq:phifd}
\end{equation}
where subscripts $w$ and $e$ refer to compass directions with index $i$
incresing eastward. A first order upwind scheme approximates the
interfacial value $\phi_w$ depending on the direction of the velocity
at the face, viz:
\begin{equation}
\phi_w = \left\{
\begin{array}{ll} \phi_i & u_w < 0, \\ \phi_{i-1} & u_w >= 0.
\end{array} \right.
\end{equation}
This choice is conditionally stable for the Euler forward time step
of Eq.~(\ref{eq:phifd}), but highly dissipative.



Better accuracy requires
a higher-order approximation to the interfacial value of $\phi$.

\subsubsection{Uniformly third order scheme}

We follow Leonard et al. \cite{utopia} in adopting a uniformly third
order approximation dependent upon both the normal and tangential
interfacial Courant numbers. It is extended here to three dimensions.

Consider the two dimensional problem of Figure \ref{fig:utopia}.

The flux at face $W$ is then
\[\begin{array}{ll}
c_x \Big\{ &
1/2(\phi_c + \phi_w) - 1/2c_x (\phi_c - \phi_w)
 - 1/6 (1 -c_x^2)(\phi_c - 2\phi_w + \phi_{ww})\\
& - 1/2c_y (\phi_w - \phi_{sw})\\
& - c_y (1/4 - 1/3c_x)(\phi_c - \phi_w - \phi_s + \phi_{sw})
 - c_y (1/4 - 1/6c_y)(\phi_{nw} - 2\phi_w + \phi_{sw})\\
& - 1/2c_z (\phi_w - \phi_{wd}) \\
& - c_z (1/4 - 1/3c_x) (\phi_c - \phi_w -\phi_{cd} + \phi_{wd})
 - c_z (1/4 - 1/6 c_y) (\phi_{wu} - 2 \phi_w + \phi_{wd}) \\
& + c_y c_z \big[
1/3(\phi_w - \phi_{wd}) - 1/3 (\phi_{sw} - \phi_{swd}) \\
& + (1/6 - 1/4c_x) (\phi_c - \phi_w -\phi_s + \phi_{sw}
    -(\phi_{cd} - \phi_{wd} - \phi_{sd} + \phi_{swd}))\\
& + (1/6 - 1/8c_y)(\phi_{nw} - 2 \phi_w + \phi_{sw} 
                - (\phi_{nwd} - 2\phi_{wd} + \phi_{swd}))\\
& + (1/6 - 1/8c_z)(\phi_{wu} - 2\phi_w + \phi_{wd}
                 - (\phi_{swu} - 2\phi_{sw} + \phi_{swd}))\big] \Big\}
\end{array}\]

\subsubsection{Gory details}

The flux-integral method of Leonard et al. assumes that within a
given cell, the order parameter is piecewise quadratic. So, for
cell $W$ we have
\begin{eqnarray}
\phi (\zeta, \eta, \theta) = \phi_W - {\scriptstyle \frac{1}{24}} 
(\phi_C + \phi_{SW} + \phi_{WW} + \phi_{NW} + \phi_{WD} + \phi_{WU}
- 6\phi_W)\\
+ {\scriptstyle \frac{1}{2}} (\phi_C - \phi_{WW}) \zeta
+ {\scriptstyle \frac{1}{2}} (\phi_C - 2\phi_{W} + \phi_{WW}) \zeta^2\\
+ {\scriptstyle \frac{1}{2}} (\phi_{NW} - \phi_{SW}) \eta
+ {\scriptstyle \frac{1}{2}} (\phi_{NW} - 2\phi_{W} + \phi_{SW}) \eta^2\\
+ {\scriptstyle \frac{1}{2}} (\phi_{WU} - \phi_{WD}) \theta
+ {\scriptstyle \frac{1}{2}} (\phi_{WU} - 2\phi_{W} + \phi_{WD}) \theta^2
\end{eqnarray}


\subsection{Lattice kinetic equation viewed as finite difference}

For binary fluid problems the second distribution $g_i (\mathbf{x};t)$,
representing the composition, obeys the evolution equation
\begin{equation}
g_i (\mathbf{x}; t + \Delta t )
= g_i^\star (\mathbf{x} - \mathbf{c}_i \Delta t; t)
\label{lke}
\end{equation}
where the star indicates the post-collision distribution. Here we consider
the distribution to be set by requiring the first moment
\begin{equation}
j_\alpha = \sum_i g_i^\star (\mathbf{x};t) c_{i\alpha} = \phi u_\alpha
\label{eq:moment1}
\end{equation}
and the second moment as
\begin{equation}
\Phi_{\alpha\beta} = \sum_i g_i^\star (\mathbf{x}; t) c_{i\alpha} c_{i\beta}
= \phi u_\alpha u_\beta + 2M\mu (\mathbf{x};t) \delta_{\alpha\beta}.
\end{equation}
Here the mobility $M$ enters through the terms related to the chemical
potential $\mu (\mathbf{x};t)$. The distribution is then set using the
reprojection
\begin{equation}
g_i^\star (\mathbf{x}; t) = \delta_{i0}\phi
+ w_i (j_\alpha u_{i\alpha} / c_s^2
+      \Phi_{\alpha\beta} Q_{i\alpha\beta} / 2c_s^4),
\end{equation}
where the $\delta_{i0}$ moves $\phi$ mostly into the non-propagating
distribution.

For a uniform velocity field $u_\alpha = (u_x, u_y, u_z)$ it is possible
to expand \ref{eq:lke} in a finite difference form which includes three
parts: the advective part in $\phi$ related to the first moment, a
dissipative contribution in $\phi$ related to the $\phi u_\alpha u_\beta$
term in the second moment, and a part related to the
 diffusion of the chemical potential.

\subsubsection{Advective terms}

Combining \ref{eq:lke} and \ref{eq:moment1} and taking the sum over
the distributions we have
\begin{equation}
\sum_i g_i (\mathbf{x};t) = \sum_i \Big( g_0^\star (\mathbf{x};t)  +
(1/c_s^2)  w_i \phi (\mathbf{x} - \mathbf{c}_i \Delta t;t)
u_\alpha c_{i\alpha} \Big).
\end{equation}
Introducing a finite finiterence notation $\phi_{ijk}^n
= \sum_{i'} g_{i'}^\star (\mathbf{x}; t)$ where the indices $i,j,k$
represent the spatial discretistation, and the superscript $n$ represents
the discrete time level we have
\begin{eqnarray}
\phi_{ijk}^{n+1}  = \phi_{ijk}^n - u_x w_1/c_s^2 \Big\{
\phi_{i+1 j k}^n - \phi_{i-1 j k}^n \\
+ 1/2(
\phi_{i+1 j+1 k}^n - \phi_{i-1 j-1 k}^n +
\phi_{i+1 j-1 k}^n - \phi_{i-1 j+1 k}^n +
\phi_{i+1 j k+1}^n - \phi_{i-1 j k-1}^n +
\phi_{i+1 j k-1}^n - \phi_{i-1 j k-+}^n)
\Big\} \\ +
u_y w_1 / c_s^2 \Big\{
\phi_{i j+1 k}^n - \phi_{i j-1 k}^n \\
+ 1/2 (
\phi_{i+1 j+1 k}^n - \phi_{i-1 j-1 k}^n +
\phi_{i-1 j+1 k}^n - \phi_{i+1 j-1 k}^n +
\phi_{i j+1 k+1}^n - \phi_{i j-1 k-1}^n +
\phi_{i j+1 k-1}^n - \phi_{i j-1 k+1}^n)
\Big\} \\ +
u_z w_1 / c_s^2 \Big\{
\phi_{i j k+1}^n - \phi_{i j k-1}^n \\
+ 1/2 (
\phi_{i+1 j k+1}^n - \phi_{i-1 j k-1}^n +
\phi_{i-1 j k+1}^n - \phi_{i+1 j k-1}^n +
\phi_{i j+1 k+1}^n - \phi_{i j-1 k-1}^n +
\phi_{i j-1 k+1}^n - \phi_{i j+1 k-1}^n) \Big\}.
\end{eqnarray}
We can regcognise here second order centred difference approximations to
the first derivative in $\phi$ in the coordinate direction in the
diagonal directions (cf \cite{sescu2008} with 'isotropy correction factor'
$\beta = 1/2$ in three dimensions).

There are a number of points to note about this. First, there are
terms in $\sum_i g_0^\star$ which are not included above, but are dealt
with below. Second, while the spatial discretisation looks ok, this is
forward in time, which would classically be undesirable. Dissipation
is required to prevent dispersive errors dominating. (Although with
general flow field, the coefficients of the finite difference form are
altered in ways which could provide some upwind bias.)

\subsubsection{Dissipative terms} 

The dissipative terms from the $\phi u_\alpha u_\beta$ term in the
second moment give
\begin{equation}
\sum_{i'} g_{i'} (\mathbf{x}; t + \Delta t) =
1/2c_s^4 \sum_{i'} w_i \phi (\mathbf{x} - \mathbf{c}_i \Delta t; t)
u_\alpha u_\beta Q_{i'\alpha\beta}.
\end{equation}
With a little effort this may be expanded to give and evolution
equation with the following difusive tendency terms:
\begin{eqnarray}
w_1/2c_s^4  \Big\{
(u_x^2 - c_s^2u^2)(\phi_{i+1 j k} - 2\phi_{ijk} + \phi_{i-1 j k}) \\+
(u_y^2 - c_s^2u^2)(\phi_{i j+1 k} - 2\phi_{ijk} + \phi_{i j-1 k}) \\+
(u_z^2 - c_s^2u^2)(\phi_{i j k+1} - 2\phi_{ijk} + \phi_{i j k-1}) \Big\}\\
+ w_2/2c_s^4 \Big\{
\big[ (u_x^2 + u_y^2) - c_s^2u^2\big]
(\phi_{i+1 j+1 k} - 2\phi_{ijk} + \phi_{i-1 j-1 k}) \\+
\big[ (u_x^2 - u_y^2) - c_s^2u^2 \big]
(\phi_{i+1 j-1 k} - 2\phi_{ijk} + \phi_{i-1 j+1 k}) \\+
\big[ (u_x^2 + u_z^2) - c_s^2u^2 \big]
(\phi_{i-1 j k-1} - 2\phi_{ijk} + \phi_{i+1 j k+1}) \\+
\big[ (u_x^2 - u_z^2) - c_s^2u^2 \big]
(\phi_{i+1 j k-1} - 2\phi_{ijk} + \phi_{i-1 j k+1}) \\+
\big[ (u_y^2 + u_z^2) - c_s^2u^2 \big]
(\phi_{i j-1 k-1} -2\phi_{ijk} + \phi_{i j+1 k+1}) \\+
\big[ (u_y^2 - u_z^2) - c_s^2u^2 \big]
(\phi_{i j+1 k-1} - 2\phi_{ijk} + \phi_{i j-1 k+1}) \Big\}.
\end{eqnarray}
It can be see that the coefficients of these diffusive terms is
related to the aspect ratio of the flow, and that the coefficents
of given terms could become negative.

\subsubsection{Chemical potential term}

Finally, the terms in the chemical potential give
\begin{equation}
\sum_{i'} g(\mathbf{x};t + \Delta t) =
\sum_{i'} w_i 2M\mu(\mathbf{x} - \mathbf{c}_ \Delta t)
\delta_{\alpha\beta} Q_{i\alpha\beta}.
\end{equation}
Noting that the contraction $\delta_{\alpha\beta} Q_{i\alpha\beta} =
-1, 0, 1$ for velocities with weights $w_0, w_1, w_2$ respectively,
we get
\begin{eqnarray}
\phi_{ijk}^{n+1} = \phi_{ijk}^n -  w_0 2M u_{ijk}/2c_s^4
+ 2M w_2/2c_s^4 \Big\{
\mu_{i+1 j+1 k} + \mu_{i-1 j-1 k} + \mu_{i+1 j-1 k} + \mu_{i-1 j+1 k} + \\
\mu_{i+1 j k+1} + \mu_{i-1 j k-1} + \mu_{i+1 j k-1} + \mu_{i-1 j k+1} +
\mu_{i j+1 k+1} + \mu_{i j-1 k-1} + \mu_{i j+1 k-1} + \mu_{i j-1 k+1} \Big\}.
\end{eqnarray}
This can be combined to give an easily recognisable finite difference
solution for the diffusion equation
\begin{eqnarray}
\phi_{ijk}^{n+1} = \phi_{ijk}^n + 2M w_2 / 2 c_s^4
\Big\{
\mu_{i+1 j+1 k} - 2\mu_{ijk} + \mu_{i-1 j-1 k} + 
\mu_{i+1 j-1 k} - 2\mu_{ijk} + \mu_{i-1 j+1 k} + \\
\mu_{i+1 j k+1} - 2\mu_{ijk} + \mu_{i-1 j k-1} +
\mu_{i+1 j k-1} - 2\mu_{ijk} + \mu_{i-1 j k+1} + \\
\mu_{i j+1 k+1} - 2\mu_{ijk} + \mu_{i j-1 k-1} + 
\mu_{i j+1 k-1} - 2\mu_{ijk} + \mu_{i j-1 k+1} \Big\}.
\end{eqnarray}
Note that this uses the twelve stencil points in $\mu$ related to
the velocities with weight $w_2$ and so should have very good isotropy
properties. The final prefactor is $M$ (check).

\subsection{User input}

\inputkey{free\_energy symmetric}

The following applies for the binary fluid problem with
compositional order parameter $\phi$ and free energy
(excluding the term in the density $\rho$):
\begin{equation}
 F[\phi] = 
\int dr \left(
{\textstyle \frac{1}{2}}A\phi^2
+ {\textstyle \frac{1}{4}}B\phi^4
+ {\textstyle \frac{1}{2}}\kappa (\mathbf{\nabla}\phi)^2 \right).
\end{equation}

This is described in some detail
by Kendon et al \cite{viv}. The first two terms represent the bulk
contribution, whereas the term in $\kappa$ penalises curvature in
the interface.

\inputkey{A} The parameter $A$. Note $A < 0$.

\inputkey{B} The parameter $B$. Note $B = -A$ for common usage,
although this is not enforced.

\inputkey{K} The parameter $\kappa$, which is positive.

The order parameter evolution is determined by a Cahn-Hilliard equation
with mobility set by

\inputkey{mobility} Sets mobility $M$ (uniform in space).

The following parameters control the initialisation of the order
parameter.

\inputkey{phi\_initialisation}

Determines how the compositional order parameter is initialised
at the start of the run. If set to \texttt{spinodal} the value
is set to $\phi_0$ as set above plus or minus a random noise, the
magnitude of which is set by the value of the \texttt{noise} key.
If set to \texttt{block} a one-dimensional profile is set up in
the $z$-direction representing two blocks of fluid with
$\phi = \pm 1$. The two interfaces are set at $z = L_z/4$ and
$z = 3L_z/4$ with the equilibrium $\tanh(z/\xi_0)$ profile having 
appropriate width. The $\phi = -1$ section is in the middle.

\inputkey{phi0}

The mean compositional order parameter roughly $-0.5 < \phi_0 < 0.5$.
The default value is zero, i.e., a symmetric 50:50 mixture by volume.

\inputkey{noise}

The magnitude of the initial fluctuations in $\phi$ used to
initiate spinodal decomposition.

\subsubsection{Binary fluid using two distributions}

\inputkey{symmetric\_lb}

This is the special case where the composition is represented
by a second LB distribution, and an appropriate lattice kinetic
equation approximating the Cahn-Hilliard equation is solved.
In this case the above parameters have the same meaning.

Note that the relaxation time for the second distribution is
related to the mobility by $tau_\phi = M\rho_0/\Delta t + 1/2$.


\subsubsection{Brazovskii}

\inputkey{free\_energy brazovskii}

This is similar to the symmetric free energy, but with one extra term
in a higher derivative of $\phi$.
\begin{equation}
 F[\phi] = 
\int dr \left(
{\textstyle \frac{1}{2}}A\phi^2
+ {\textstyle \frac{1}{4}}B\phi^4
+ {\textstyle \frac{1}{2}}\kappa (\mathbf{\nabla}\phi)^2
+ {\textstyle \frac{1}{2}} C (\nabla^2 \phi)^2 \right).
\end{equation}

The parameters now include $C$. For $A<0$, phase separation occurs
with a result depending on $\kappa$: ones get two symmetric phases
for $\kappa >0$ (cf. symmetric) or a lamellar phase for
$\kappa < 0$.

\inputkey{A} Bulk parameter $A < 0$.

\inputkey{B} Bulk parameter $B = -A$.

\inputkey{K} Negative for lamellar phase.

\inputkey{C} Positive.

Other parameters, including the mobility, are set as for the
symmetric free energy (see above).



