\chapterauthor{Raphaël Couturier}{Femto-ST Institute, University of Franche-Comte}

\chapter{Introduction to Cuda}
\label{chapter2}

\section{Introduction}
\label{ch2:intro}

In this chapter  we give some simple examples of Cuda  programming.  The goal is
not to provide an exhaustive presentation of all the functionalities of Cuda but
rather to give some basic elements. Of  course, readers that do not know Cuda are
invited  to read  other  books that  are  specialized on  Cuda programming  (for
example: \cite{ch2:Sanders:2010:CEI}).


\section{First example}
\label{ch2:1ex}

This first example is  intented to show how to build a  very simple program with
Cuda.  Its goal  is to perform the sum  of two arrays and put the  result into a
third array.  A Cuda program consists in  a C code which calls Cuda kernels that
are executed on a GPU. The listing of this code is in Listing~\ref{ch2:lst:ex1}.


As GPUs have  their own memory, the first step consists  in allocating memory on
the GPU.  A call to  \texttt{cudaMalloc}\index{Cuda~functions!cudaMalloc} allows
to allocate memory on the GPU. The first parameter of this function is a pointer
on a memory on the device (i.e. the GPU). In this example, \texttt{d\_} is added
on each variable allocated  on the GPU, meaning this variable is  on the GPU. The
second parameter represents the size of the allocated variables, this size is expressed in
bits.

In this example, we  want to compare the execution time of  the additions of two
arrays in  CPU and  GPU. So  for both these  operations, a  timer is  created to
measure the  time. Cuda proposes to  manipulate timers quite  easily.  The first
step is to create the timer\index{Cuda~functions!timer}, then to start it and at
the end to stop it. For each of these operations a dedicated function is used.

In  order to  compute  the same  sum  with a  GPU, the  first  step consists  in
transferring the data from the CPU (considered as the host with Cuda) to the GPU
(considered as the  device with Cuda).  A call  to \texttt{cudaMemcpy} allows to
copy the content of an array allocated in the host to the device when the fourth
parameter                                 is                                 set
to  \texttt{cudaMemcpyHostToDevice}\index{Cuda~functions!cudaMemcpy}.  The first
parameter of the function is the  destination array, the second is the
source  array and  the third  is the  number of  elements to  copy  (expressed in
bytes).

Now the  GPU contains the  data needed to  perform the addition.   In sequential
programming, such  addition is  achieved out  with a loop  on all  the elements.
With a GPU,  it is possible to perform  the addition of all the  elements of the
two  arrays in  parallel (if  the number  of blocks  and threads  per  blocks is
sufficient).   In Listing\ref{ch2:lst:ex1}  at the  beginning, a  simple kernel,
called \texttt{addition} is defined to  compute in parallel the summation of the
two     arrays.      With     Cuda,     a     kernel     starts     with     the
keyword   \texttt{\_\_global\_\_}   \index{Cuda~keywords!\_\_shared\_\_}   which
indicates that this kernel can be called from the C code.  The first instruction
in this kernel is used to compute the variable \texttt{tid} which represents the
thread index.   This thread index\index{thread  index} is computed  according to
the           values            of           the           block           index
(called  \texttt{blockIdx} \index{Cuda~keywords!blockIdx}  in Cuda)  and  of the
thread   index   (called   \texttt{blockIdx}\index{Cuda~keywords!threadIdx}   in
Cuda). Blocks of threads and thread  indexes can be decomposed into 1 dimension,
2 dimensions or  3 dimensions.  According to the  dimension of manipulated data,
the appropriate dimension  can be useful. In our example,  only one dimension is
used.   Then using notation  \texttt{.x} we  can access  to the  first dimension
(\texttt{.y}  and \texttt{.z}  respectively allow to access  to the  second and
third dimension).   The variable \texttt{blockDim}\index{Cuda~keywords!blockDim}
gives the size of each block.



\lstinputlisting[label=ch2:lst:ex1,caption=A simple example]{Chapters/chapter2/ex1.cu}

\section{Second example: using CUBLAS}
\label{ch2:2ex}

The Basic Linear Algebra Subprograms  (BLAS) allows programmers to use efficient
routines  that are  often  required. Those  routines  are heavily  used in  many
scientific applications  and are optimized for  vector operations, matrix-vector
operations                           and                           matrix-matrix
operations~\cite{ch2:journals/ijhpca/Dongarra02}. Some  of those operations seem
to be  easy to  implement with Cuda.   Nevertheless, as  soon as a  reduction is
needed, implementing an efficient reduction routine with Cuda is far from being
simple. Roughly speaking, a reduction operation\index{reduction~operation} is an
operation  which combines  all the  elements of  an array  and extracts  a number
computed with all the  elements. For example, a sum, a maximum  or a dot product
are reduction operations.

In this second example, we consider that  we have two vectors $A$ and $B$. First
of all, we want to compute the sum  of both vectors in a vector $C$. Then we want
to compute the  scalar product between $1/C$ and $1/A$. This  is just an example
which has no direct interest except to show how to program it with Cuda.

Listing~\ref{ch2:lst:ex2} shows this example with Cuda. The first kernel for the
addition  of two  arrays  is exactly  the same  as  the one  described in  the
previous example.

The  kernel  to  compute the  opposite  of  the  elements  of  an array  is  very
simple. For  each thread index,  the inverse of  the array replaces  the initial
array.

In the main function,  the beginning is very similar to the  one in the previous
example.  First,  the user is  askef to define  the number of elements.   Then a
call  to \texttt{cublasCreate}  allows  to initialize  the  cublas library.   It
creates a handle. Then all the arrays  are allocated in the host and the device,
as in the  previous example.  Both arrays $A$ and $B$  are initialized.  The CPU
computation is performed  and the time for this CPU  computation is measured. In
order to  compute the same result  on the GPU, first  of all, data  from the CPU
need to be  copied into the memory of  the GPU. For that, it is  possible to use
cublas   function   \texttt{cublasSetVector}.    This   function   has   several
arguments. More precisely, the first  argument represents the number of elements
to transfer, the second arguments is the size of each element, the third element
represents the source  of the array to  transfer (in the GPU), the  fourth is an
offset between each element of the source  (usually this value is set to 1), the
fifth is  the destination (in the  GPU) and the  last is an offset  between each
element  of the  destination. Then  we call  the kernel  \texttt{addition} which
computes the  sum of all elements  of arrays $A$ and  $B$.  The \texttt{inverse}
kernel  is called twice,  once to  inverse elements  of array  $C$ and  once for
$A$. Finally,  we call the  function \texttt{cublasDdot} which computes  the dot
product  of two  vectors.   To use  this  routine, we  must  specify the  handle
initialized by  Cuda, the number  of elements to  consider, then each  vector is
followed by the offset between every  element.  After the GPU computation, it is
possible to check that both computation produce the same result.

\lstinputlisting[label=ch2:lst:ex2,caption=A simple example with cublas]{Chapters/chapter2/ex2.cu}

\section{Third example: matrix-matrix multiplication}
\label{ch2:3ex}



Matrix-matrix multiplication is an operation  which is quite easy to parallelize
with a GPU. If we consider that  a matrix is represented using a two dimensional
array, $A[i][j]$ represents the element of  the $i^{th}$ row and of the $j^{th}$
column. In  many cases, it is  easier to manipulate a  1D array instead  of a 2D
array.   With Cuda,  even if  it is  possible to  manipulate 2D  arrays,  in the
following we present an example based on a 1D array. For the sake of simplicity,
we  consider we  have  a square  matrix of  size  \texttt{size}.  So  with a  1D
array,  \texttt{A[i*size+j]} allows  us to  have access  to the  element  of the
$i^{th}$ row and of the $j^{th}$ column.

With  a sequential  programming, the  matrix multiplication  is  performed using
three loops. We assume that $A$, $B$  represent two square matrices and the
result   of    the   multiplication    of   $A   \times    B$   is    $C$.   The
element \texttt{C[i*size+j]} is computed as follows:
\begin{equation}
C[i*size+j]=\sum_{k=0}^{size-1} A[i*size+k]*B[k*size+j];
\end{equation}

In Listing~\ref{ch2:lst:ex3},  the CPU computation  is performed using  3 loops,
one  for $i$,  one for  $j$  and one  for $k$.   In  order to  perform the  same
computation on a  GPU, a naive solution consists in  considering that the matrix
$C$ is split into  2 dimensional blocks.  The size of each  block must be chosen
such as the number of threads per block is inferior to $1,024$.


In Listing~\ref{ch2:lst:ex3},  we consider that  a block contains 16  threads in
each   dimension,  the   variable  \texttt{width}   is  used   for   that.   The
variable \texttt{nbTh} represents the number of threads per block. So, to be able
to compute the matrix-matrix product on a GPU, each block of threads is assigned
to compute the result  of the product for the elements of  this block.  The main
part of the code is quite similar to the previous code.  Arrays are allocated in
the  CPU and  the GPU.   Matrices $A$  and $B$  are randomly  initialized.  Then
arrays are  transferred inside the  GPU memory with call  to \texttt{cudaMemcpy}.
So the first step for each thread of a block is to compute the corresponding row
and   column.    With   a    2   dimensional   decomposition,   \texttt{int   i=
blockIdx.y*blockDim.y+ threadIdx.y;} allows us to compute the corresponding line
and  \texttt{int  j=   blockIdx.x*blockDim.x+  threadIdx.x;}  the  corresponding
column. Then each  thread has to compute the  sum of the product of  the line of
$A$   by   the  column   of   $B$.    In  order   to   use   a  register,   the
kernel  \texttt{matmul}  uses a  variable  called  \texttt{sum}  to compute  the
sum. Then the result is set into  the matrix at the right place. The computation
of  CPU matrix-matrix multiplication  is performed  as described  previously.  A
timer measures  the time.   In order to  use 2 dimensional  blocks, \texttt{dim3
dimGrid(size/width,size/width);} allows us  to create \texttt{size/width} blocks
in each  dimension.  Likewise,  \texttt{dim3 dimBlock(width,width);} is  used to
create \texttt{width} thread  in each dimension. After that,  the kernel for the
matrix  multiplication is  called. At  the end  of the  listing, the  matrix $C$
computed by the GPU is transferred back  into the CPU and we check if both matrices
C computed by the CPU and the GPU are identical with a precision of $10^{-4}$.


With $1,024  \times 1,024$ matrices,  on a C2070M  Tesla card, this  code takes
$37.68$ms to perform the multiplication. With an Intel Xeon E31245 at $3.30$GHz, it
takes $2465$ms  without any parallelization (using only  one core). Consequently
the speed up  between the CPU and GPU  version is about $65$ which  is very good
regarding the difficulty of parallelizing this code.

\lstinputlisting[label=ch2:lst:ex3,caption=simple Matrix-matrix multiplication with cuda]{Chapters/chapter2/ex3.cu}

\section{Conclusion}
In this chapter, three simple Cuda examples have been  presented. They are
quite  simple. As we  cannot  present  all the  possibilities  of  the  Cuda
programming, interested  readers  are  invited  to  consult  Cuda  programming
introduction books if some issues regarding the Cuda programming are not clear.

\putbib[Chapters/chapter2/biblio]

