%%%%%%%%%%%%%%%%%%%%%%%%%%%%%%%%%%%%%%%%%%%%%%%%%%%%%%%%%%%%%%%%%%%%%%%%%%%%%%%
%
%  colloid.tex
%
%  Largely colloid hydrodynamics and colloid-colloid interactions
%
%  Edinburgh Soft Matter and Statistical Physics Group and
%  Edinburgh Parallel Computing Centre
%
%  Contributing authors:
%  Kevin Stratford (kevin@epcc.ed.ac.uk)
%
%%%%%%%%%%%%%%%%%%%%%%%%%%%%%%%%%%%%%%%%%%%%%%%%%%%%%%%%%%%%%%%%%%%%%%%%%%%%%%%

\section{Colloid Dynamics}

\label{section:colloids}


\subsection{Hydrodynamics}

\label{section-colloid-hydrodynamics}

Having computed the total force and torque on an individual colloid
by the process of bounce-back-on-links,
is is possible to update the the linear velocities
\begin{equation}
m_0 \mathbf{U}(t + \Delta t) = m_0 \mathbf{U}(t) + \Delta t \mathbf{F} (t),
\end{equation}
where the mass of the colloid is related to the input radius by
$m_0 = (4/3)\pi\rho_0 a_0^3$ and the angular
velocity
\begin{equation}
I_0 \mathbf{\Omega} (t + \Delta t) = I_0 \mathbf{\Omega}(t) + \Delta t
\mathbf{T}(t),
\end{equation}
where the moment of inertia is $I_0 = (2/5) m_0 a_0^2$.
However, this explicit update is generally found to have poor stability
properties \cite{l94b, nguyen-ladd2002}. The alternative is to use a
velocity update
which is implicit \cite{heemels,nguyen-ladd2002}.


The total force and torque on a colloid can be split into a velocity-dependent
and a velocity-independent part by combining Equations~(\ref{eq-colloid-ub})
and~(\ref{eq-colloid-fb}) to eliminate the boundary velocity $\mathbf{u}_b$.
In this way, one can write
\begin{equation}
\left( \begin{array}{rr} \mathbf{F} \\ \mathbf{T}
\end{array} \right) =
\left( \begin{array}{rr} \mathbf{F}_0 \\ \mathbf{T}_0
\end{array} \right) -
\left( \begin{array}{rr}
\mathbf{\boldsymbol{\zeta}}^{FU} & \mathbf{\boldsymbol{\zeta}}^{F\Omega} \\
\mathbf{\boldsymbol{\zeta}}^{TU} & \mathbf{\boldsymbol{\zeta}}^{T\Omega}
\end{array} \right)
\left( \begin{array}{rr} \mathbf{U} \\ \mathbf{\Omega}
\end{array} \right).
\end{equation}
%%
Here, the velocity independent parts of the force and the torque
(appropriate for a colloid at rest) are
\begin{eqnarray}
\mathbf{F}_0(t + {\scriptstyle\frac{1}{2}}\Delta t) =
\frac{2\Delta x^3}{\Delta t} \sum_b \big[ f_b^\ast(\mathbf{r}; t)
- f_{b'}^\ast(\mathbf{r} + \mathbf{c}_i \Delta t; t) \big] \mathbf{c}_b,
\\
\mathbf{T}_0(t + {\scriptstyle\frac{1}{2}}\Delta t) =
\frac{2\Delta x^3}{\Delta t} \sum_b \big[ f_b^\ast(\mathbf{r}; t)
- f_{b'}^\ast(\mathbf{r} + \mathbf{c}_i \Delta t; t) \big]
(\mathbf{r}_b\times \mathbf{c}_b),
\end{eqnarray}
and the matrices $\boldsymbol{\zeta}$ are interpreted as drag
coefficients and can be written as
\begin{eqnarray}
\mathbf{\boldsymbol{\zeta}}^{FU} &=& \frac{4\rho_0 \Delta x^3}{c_s^2 \Delta t}
\sum_b w_{c_b} \mathbf{c}_b \mathbf{c}_b,
\\
\mathbf{\boldsymbol{\zeta}}^{F\Omega} &=&
\frac{4\rho_0 \Delta x^3}{c_s^2 \Delta t}
\sum_b w_{c_b} \mathbf{c}_b (\mathbf{r}_b \times \mathbf{c}_b),
\\
\mathbf{\boldsymbol{\zeta}}^{TU} &=& \frac{4\rho_0 \Delta x^3}{c_s^2 \Delta t}
\sum_b w_{c_b} (\mathbf{r}_b \times \mathbf{c}_b) \mathbf{c}_b,
\\
\mathbf{\boldsymbol{\zeta}}^{T\Omega} &=&
\frac{4\rho_0 \Delta x^3}{c_s^2 \Delta t}
\sum_b w_{c_b} (\mathbf{r}_b \times \mathbf{c}_b)
(\mathbf{r}_b \times \mathbf{c}_b).
\end{eqnarray}

The velocity updates can now be rewritten using an
implicit (or self-consistent) approach, where the velocity
at the new time step is used in computing the velocity-dependent
terms:
\begin{equation}
\label{eq-colloid-implicit-update}
\left( \begin{array}{rr}
m_0 \mathbf{U}(t + \Delta t) \\ I_0\mathbf{T}(t + \Delta t)
\end{array} \right) =
\left( \begin{array}{rr} m_0 \mathbf{U}(t) \\ I_0\mathbf{T}(t)
\end{array} \right) +
%%
\Delta t
\bigg[
\left( \begin{array}{rr} \mathbf{F}_0 \\ \mathbf{T}_0
\end{array} \right) -
\left( \begin{array}{rr}
\mathbf{\boldsymbol{\zeta}}^{FU} & \mathbf{\boldsymbol{\zeta}}^{F\Omega} \\
\mathbf{\boldsymbol{\zeta}}^{TU} & \mathbf{\boldsymbol{\zeta}}^{T\Omega}
\end{array} \right)
\left( \begin{array}{rr}
\mathbf{U}(t + \Delta t) \\ \mathbf{\Omega}(t + \Delta t)
\end{array} \right)
\bigg].
\end{equation}


If the particle has a symmetric distribution of boundary links (the
special case where the centre of mass is on a symmetry point of the
lattice) the $\boldsymbol{\zeta}^{FU}$ and
$\boldsymbol{\zeta}^{T\Omega}$ matrices are
diagonal, while the $\boldsymbol{\zeta}^{F\Omega}$
and $\boldsymbol{\zeta}^{TU}$ matrices are zero. In general, however,
this is not the case, and the full drag matrix must be used to form
the update of the dynamic quantities in which case
Equation~\ref{eq-colloid-implicit-update} is solved
as a 6$\times$6 matrix inversion.

Finally, the position of the particle can be updated: an Euler
forward step is taken using the mean of the old and updated velocity
\begin{equation}
\label{eq-colloid-position-update}
\mathbf{r} (t + \Delta t) = \mathbf{r} (t) + (1/2)\Delta t
\big[ \mathbf{U}(t) + \mathbf{U}(t + \Delta t) \big].
\end{equation}


\subsubsection{Rotational motions}

For a number of applications it is useful to update not only
position, as in Equation~\ref{eq-colloid-position-update},
but also orientation. This is relevant for magnetic
dipoles, swimming directions,  and so on. A convenient way to
implement the required rotations is by considering quaternions.

A quaternion can be thought of as an extended complex number
\begin{equation}
q = w + xi + yj + zk
\end{equation}
where $i^2 = j^2 = k^2 = -1$ and $ijk = -1$. The norm, or length
of a quaternion is
\begin{equation}
N(q) = (w^2 + x^2 + y^2 + z^2)^{1/2}.
\end{equation}
Addition and subtraction of quaternions proceeds as normal, but
care is required for multiplication, which is associative but not
commutative. If the quaternion is written as $q = (w, {\bf v})$ where
$\bf{v}$ is a normal 3-vector, then multiplication can be written
\begin{equation}
q_1 q_2 = (w_1 w_2 - {\bf v}_1 . {\bf v}_2, w_1 {\bf v}_2 + w_2 {\bf v}_1
+ {\bf v}_1 \times {\bf v}_2),
\end{equation}
where the usual scalar and vector products are used. Note that the
conjugate of a quaternion $q^\star = (w, -{\bf v})$ so that
$qq^\star = N^2(q)$.
It can be shown that a rotation of an arbitrary vector though an
angle $\theta$ around the unit axis of rotation $\hat{\bf w}$ can
be represented by the quaternion
\begin{equation}
q = (\cos(\theta/2), \hat{\bf w} \sin(\theta/2))
\end{equation}
which ensures that $N(q) = 1$. If the arbitrary vector is (0, {\bf v}),
then the rotated vector ${\bf v}^{'}$ can be written as
\begin{equation}
(0, {\bf v}^{'}) = q (0, {\bf v}) q^\star
\end{equation}
which can be expanded as
\begin{equation}
{\bf v}^{'} = (1 - \cos\theta)({\bf v}.{\hat{\bf w}) \hat{\bf w} +
{\bf v}\cos\theta} + (\hat{\bf w} \times {\bf v})\sin\theta.  
\end{equation}
This equation is used to update appropriate vector quantities at
each time step by a rotation
$\mathbf{w} = \Delta t \mathbf{\Omega}(t +\Delta t)$,
where the angular velocity is that determined by the implicit
procedure described in Section~\ref{section-colloid-hydrodynamics}.

\subsubsection{Changes in particle shape}

One consequence is that the discrete shape of the particle fluctuates
as the colloid centre moves relative to the
lattice. This means that the size of the particle as seen by the
fluid changes, despite the fact that the input radius of the colloid
is fixed.  However, these fluctuations are small for input radii
greater than about 5 lattice units \cite{l96a}. Furthermore, the
fluctuations are greater for some input radii than others
(leading to preferred values of $a_0$ 1.25, 2.33, and so on).

In the original approach of Ladd \cite{l96a}, changes in the map of
boundary links are accommodated by the internal fluid. If the particle
shape changes so that an internal node is exposed, the internal fluid
at that site rejoins the fluid proper with the solid body momentum
(assuming the internal fluid is relaxed to solid-body rotation).
If a fluid node is covered by particle movement, then it simply
joins the internal fluid, and will relax to solid-body rotation.
Without internal fluid, changes in particle shape in which fluid nodes
are either covered or uncovered must be accompanied by the removal or
addition of fluid with appropriate properties at the nodes in question.
It is important that this is done in a way which minimises the
perturbation to the fluid flow around the particle.

If a fluid node at $\mathbf{r}$ is covered by the movement of a
solid particle, the fluid loses a mass $\rho(\mathbf{r}) \Delta x^3$,
of which an excess $\Delta M_c = (\rho(\mathbf{r}) - \rho_0)\Delta x^3$
must be replaced explicitly so that the overall mean fluid density is
unchanged. At the same time, the colloid must assume the momentum lost
by the fluid
$\Delta x^3 \sum_i f_i(\mathbf{r}) \mathbf{c}_i$.

Similarly, when a fluid node is exposed by particle movement,
fluid must be replaced with the appropriate distribution, density
and velocity. If the
new density is $\rho$ (to be determined by some method) then
the fluid gains an excess of mass $\Delta M_u = (\rho - \rho_0)
\Delta x^3$ which must be balanced elsewhere. If the new fluid
is assumed to have velocity $\mathbf{u}$ then the particle must
give up momentum $\Delta x^3 \rho\mathbf{u}$.

Following Nguyen and Ladd \cite{nguyen-ladd2002}, these changes owing
to change in particle shape are implemented by adding a small
correction to the bounce-back at the following time step.
The excess mass from a covered fluid site is redistributed over all the
boundary nodes by redefined the momentum transfer at bounce-back
\begin{equation}
f_{b'}(\mathbf{r}; t + \Delta t) = f_b^\ast (\mathbf{r}; t)
- \frac{2 w_{c_b} \rho_0 \mathbf{u}_b . \mathbf{c}_b}{c_s^2}
+ \frac{w_{c_b} \rho_0\Delta M_c }{A},
\label{eq:q1}
\end{equation}
where $A = \rho_0 \Delta x^3 \sum_b w_{c_b}$. The accompanying
force on the particle owing to the change in shape is then
that lost by the fluid plus the contributions from the final term
in (\ref{eq:q1}) summed over all the boundary links
\begin{equation}
\Delta \mathbf{F}_c = \frac{\Delta x^3}{\Delta t}\sum_{i} f_i \mathbf{c}_i
+ \frac{\Delta x^3}{\Delta t} \sum_b \frac{w_{c_b} \rho_0 \Delta M_c}{A}
\mathbf{c}_b.
\label{eq:q2}
\end{equation}

Note that for a closed surface,
$\sum_b w_{c_b}\mathbf{c}_b = 0$, and the second term on the right-hand
side of Eq.~(\ref{eq:q2}) is zero, i.e., the redistribution of mass
does not contribute to the net force and torque on the particle.
The corresponding change in torque is
\begin{equation}
\Delta \mathbf{T}_c = \frac{\Delta x^3}{\Delta t} \mathbf{r}_c
\times \sum_i f_i \mathbf{c}_i
+ \frac{\Delta x^3}{\Delta t}
\sum_b \frac{w_{c_b}\rho_0 \Delta M_c}{A} \mathbf{r}_b \times \mathbf{c}_b,
\label{eq:q2a}
\end{equation}
where $\mathbf{r}_c$ is the boundary vector at the position where the
fluid has been removed. Again, for a closed surface, the
redistribution of mass does not contribute to the net torque
on the particle.

Likewise, the contribution to the bounce-back from newly uncovered nodes
is
\begin{equation}
- \frac{w_{c_b} \rho_0 \Delta M_u}{A}
\label{eq:q3}
\end{equation}
leading to a change in force of
\begin{equation}
\Delta \mathbf{F}_u = -\frac{\Delta x^3 \rho(\mathbf{r})
\mathbf{u}(\mathbf{r})}{\Delta t} - \frac{\Delta x^3}{\Delta t} \sum_b
\frac{w_{c_b} \rho_0 \Delta M_u}{A} \mathbf{c}_b,
\label{eq:q4}
\end{equation}
with a corresponding change in the torque.
If more than one lattice node is either covered or uncovered by
the movement of the particle, then these contributions add in a simple
way. The overall contributions can be added to the
velocity-independent terms $\mathbf{F}_0$ and $\mathbf{T}_0$ appearing in
Equation~\ref{eq-colloid-implicit-update}.



\subsubsection{Particles near contact}

Particles near contact may not possess a full set of boundary
links. This leads to a potential
non-conservation of mass associated with the particle motion
which must be corrected.

Again following Nguyen and Ladd \cite{nguyen-ladd2002}, mass conservation is
enforced explicitly using the following procedure. There is a mass
transfer associated with the bounce-back at each link of
$(2w_{c_b}\rho_0 \mathbf{u}_b . \mathbf{c}_b / c_s^2) \Delta x^3$.
The net mass transport into the particle is the sum of this
quantity over all the links, and can be written
\begin{equation}
\Delta M_s = - \frac{2\Delta x^3 \rho_0}{c_s^2}
\Big[ \mathbf{U}.\sum_b w_{c_b} \mathbf{c}_b +
  \mathbf{\Omega} . \sum_b w_{c_b} \mathbf{r}_b \times \mathbf{c}_b \Big].
\end{equation}
For any particle with a full set of boundary links, both
$\sum_b w_{c_b} \mathbf{c}_b$
and $\sum_b w_{c_b} \mathbf{r}_b \times \mathbf{c}_b$ are zero.
However, if the set of boundary links is incomplete
$\Delta M_s$ is not zero and
a correction is required. This is again made by redistributing the
excess or deficit of mass over the other existing boundary nodes.
The additional contribution to the bounce-back here is
$-w_{c_b} \rho_0 \Delta M_s/A$, where
$A = \rho_0 \Delta x^3 \sum_b w_{c_b}$ (cf.\ Equation~\ref{eq:q1}).

This contribution
now adds to the velocity-dependent part of the force and torque and
leads to a slightly different form of the drag matrices
\begin{eqnarray}
\boldsymbol{\zeta}^{FU} &=& \frac{-2\rho_0 \Delta x^3}{c_s^2 \Delta t}
\sum_b w_{c_b} (\mathbf{c}_b - \overline{\mathbf{c}_b}) \mathbf{c}_b
\\
\boldsymbol{\zeta}^{F\Omega} &=& \frac{-2\rho_0 \Delta x^3}{c_s^2 \Delta t}
\sum_b w_{c_b} \mathbf{c}_b (\mathbf{r}_b \times \mathbf{c}_b -
\overline{\mathbf{r}_b\times\mathbf{c}_b}),
\\
\boldsymbol{\zeta}^{TU} &=& \frac{-2\rho_0 \Delta x^3}{c_s^2 \Delta t}
\sum_b w_{c_b} (\mathbf{r}_b\times\mathbf{c}_b)
(\mathbf{c}_b - \overline{\mathbf{c}_b}),
\\
\boldsymbol{\zeta}^{T\Omega} &=& \frac{-2\rho_0 \Delta x^3}{c_s^2 \Delta t}
\sum_b w_{c_b} (\mathbf{r}_b\times\mathbf{c}_b)
(\mathbf{r}_b\times\mathbf{c}_b - \overline{\mathbf{r}_b\times\mathbf{c}_b}).
\end{eqnarray}
The mean quantities
\begin{equation}
\overline{\mathbf{c}_b} = \frac{\sum_b w_{c_b} \mathbf{c}_b}{ \sum_b w_{c_b}}
\end{equation}
and
\begin{equation}
\overline{\mathbf{r}_b\times\mathbf{c}_b} =
\frac{\sum_b w_{c_b} \mathbf{r}_b\times\mathbf{c}_b}{\sum_b w_{c_b}}.
\end{equation}
A little algebra will show that both occurrences of $\mathbf{c}_b$ and/or
$\mathbf{r}_b\times\mathbf{c}_b$ in the definition of the drag matrices
can be replaced by their deviation from the mean. This provides a
convenient way to compute the drag matrices (maintaining symmetry)
and computing the correct force and torque on the particle when close
to contact.


\subsection{Colloid-Colloid interactions}

Surface-surface separation between two colloids is denoted
$h = r_{12} - a_1 - a_2$ where $r_{12}$ is the centre-centre
separation, and is based on the hydrodynamic radius $a_h$.
For ease of notation all references to $a$ in this section refer to
the hydrodynamic radius $a_h$.


\subsubsection{Lubrication corrections}

Hydrodynamic lubrication interactions between two particles may
not be fully resolved on the lattice if the particles are close
to contact. The degree to which this is the case depends on the
size of the particles, and their exact discrete position on the
lattice. There will be a cut-off separation $h_c$ associated
with corrections which must be determined by calibration
\cite{nguyen-ladd2002}; the cut-off is different for the
different contributions to the interaction.

The force and torque between the pair may be corrected
by adding back a contribution taken from the known theoretical
expression for two spheres. Two contributions are handled:

\begin{enumerate}
\item {\bf Normal force}

The appropriate correction term between colloids with hydrodynamic
radii $a_1$ and $a_2$, and having separation $h < h_c$, is
\begin{equation}
\mathbf{f}_{12} = -6\pi\eta \frac{ a_1^2 a_2^2}{(a_1 + a_2)^2}
\left[ \frac{1}{h} - \frac{1}{h_c} \right]
\hat{\mathbf{r}}_{12}. ( \mathbf{U}_1 - \mathbf{U}_2)\,
\hat{\mathbf{r}}_{12}.
\end{equation}

\item
{\bf Tangential force}

The corresponding tangential component is a somewhat complex
expression:
\begin{equation}
\mathbf{f}_{12} =
-(24/15) \pi\eta A
\big[
(\mathbf{U}_1 - \mathbf{U}_2)
-  \hat{\mathbf{r}}_{12} . (\mathbf{U}_1 - \mathbf{U}_2) \hat{\mathbf{r}}_{12}
\big],
\end{equation}
where
\begin{equation}
A = \frac{a_1 a_2 (2a_1^2 + a_1 a_2 + 2a_2^2)}{(a_1 + a_2)^3}
\left[
\ln\!\left(\frac{a_1 + a_2}{2h}\right)
- \ln\!\left(\frac{a_1 + a_2}{2h_c}\right)
\right].
\nonumber
\end{equation}
\end{enumerate}


In the presence of isothermal fluctuations, there is an additional
consideration related to the dissipation related to the correction
itself. This adds an additional random force with a variance
$(2k_BT |\mathbf{f}_{12}|)^{1/2}$ to each lubrication component.


We do not implement the collective implicit update required for
stability when many particles are in mutual lubrication contact
\cite{nguyen-ladd2002}. This treats the velocity-dependent
corrections as effectively velocity independent and not as part
of the implicit update.

This means that mutual contact must be prevented
via a repulsive potential of sufficient strength to keep particles
from close approach (the update may become unstable if not). For
many purposes it may then be preferable to ignore the lubrication
corrections completely. Long-range lubrication ($h > h_c$) will
still be well represented by the lattice fluid.


\subsubsection{Hard sphere interaction}

Particles do not overlap, i.e., there is always
a hard-sphere interaction which is a function of the separation $h$:
\begin{equation}
v^{hs}(h) = \left\{
\begin{array}{ll}
\infty & h \leq 0,\\
0   & h > 0.
\end{array} \right.
\end{equation}
The hard-sphere interaction does not give rise to a force, but gives rise
to program termination if $h<0$.


\subsubsection{Soft-sphere interaction}

\label{section-colloids-soft-sphere}

A simple soft-sphere interaction is available, the basic form
of which is:
\begin{equation}
v^{ss}(r) = \epsilon (\sigma / r)^{\nu},
\end{equation}
where $\epsilon$ sets the energy scale, and $\sigma$ sets the
characteristic width. The steepness of the potential is set by
the exponent $\nu (> 0)$.

To prevent the need for calculation of long-range interactions,
the soft-sphere potential is truncated in a ``cut-and-shift''
approach. This is done in such a way as to smoothly match
both the potential and the force at the cut-off distance
$r_c$. For $r < r_c$ the potential is then
\begin{equation}
v^{ss}(r) - v^{ss}(r_c) - (r - r_c) \left.\frac{d v^{ss}}{dr}\right|_{r=r_c}.
\label{eq_ss_shift}
\end{equation}
Clearly, this potential is not exactly what we first thought of as
soft-sphere. Matching the potential smoothly ensures conservation
of energy, while matching the force smoothly prevents potential
instabilities in the molecular dynamics update.

The soft-sphere potential can be useful as a mechanism to
keep the particles from touching (which risks hard-sphere interactions),
in which case is is possible to compute the interaction as a function
of the separation $h$, instead of $r$. The force between
two particles is computed via the derivative of equation~\ref{eq_ss_shift}
with respect to $r$:
\begin{equation}
\mathbf{F}_{ij}(r) = -\left\{\frac{d v^{ss}(r)}{dr} -
\left. \frac{d v^{ss}(r)}{dr}\right|_{r=r_c} \right\} \mathbf{\hat{r}}_{ij}
\end{equation}
where $\mathbf{\hat{r}}_{ij}$ is the unit vector joining the centre of
particle $i$ to the centre of $j$.


\subsubsection{Lennard-Jones interaction}

\label{section-colloids-lennard-jones}

A cut-and-shifted Lennard-Jones interaction is available based on the
centre-centre separtion of pairs. The basic form
of the potential is
\begin{equation}
v^{lj}(r) = 4\epsilon
\left[ \left(\frac{\sigma}{r}\right)^{12} - 
\left( \frac{\sigma}{r} \right)^6 \right]
\end{equation}
where $\epsilon$ sets the energy scale, and $\sigma$ is a characteristic
width. The cut-and-shift of both the potential and the associated force
is as described in the previous section.

If using the Lennard-Jones potential, some care may be required to
ensure that no hard-sphere overlaps occur at the level of the
hydrodynamic radius $a_h$.

\subsubsection{Dipole-dipole interactions}

For particles with magnetic dipoles $\mathbf{m}_i$, long-range
dipole-dipole interactions may be computed using a standard
Ewald sum for a periodic system
(e.g., \cite{allen-tildesley}, \cite{rapaport1995}).
We describe this here only briefly.

We consider particles with uniform dipoles
$\mathbf{m}_i = \mu \mathbf{s}_i$ where $\mu$ is the dipole
strength and $\mathbf{s}_i$ is
a unit vector describing the orientation of the dipole. The
basic dipole-dipole interaction is then
\begin{equation}
v^{dd} (r) = \frac{\mu_0 \mu^2}{4\pi}
\big( \mathbf{s}_i . \mathbf{s}_j
- 3(\mathbf{s}_i.\mathbf{r}_{ij})( \mathbf{s}_j .\mathbf{r}_{ij})
\big),
\end{equation}
where $\mu_0$ is the permeability. The Ewald sum decomposes this
into a short-range part contributing forces and torques computed
in real space and reciprocal-space part computed as a sum over
reciprocal space vectors. The choice of the real-space cut-off
$r_c$ determines the number of reciprocal space vectors required
in the sum.

A relevant parameter for magnetic colloid systems is the
dipolar coupling constant which measures the strength of
the dipole-dipole interaction compared with the thermal
energy $k_bT$ (which may be introduced via isothermal
fluctuations in the fluid). If we take $\mu_0 / 4\pi$ to be
unity in lattice units, the the dipolar coupling constant
is
\begin{equation}
\lambda = \mu^2 / 8k_bT a^3.
\end{equation} 
\subsection{External Forces}

\subsubsection{Gravity}

For the purposes of sedimentation and other driven motions, the code
allows a uniform gravitational acceleration on colloids which
may be written as $\mathbf{g} = (g_x, g_y, g_z)$. We will consider the
simple situation where $\mathbf{g} = (0, 0, -g)$, i.e., the
negative $z$-direction is ``downwards.''

A colloid of density $\rho$ sedimenting in this gravitational field
experiences a net buoyancy force density $(\rho-\rho_0)g = \Delta\rho g$,
where $\rho_0$ is the fluid density. (If the colloid and fluid density
are equal, one can still think in the limit that $\Delta\rho \rightarrow 0$
and $g \rightarrow \infty$, but where the product $\Delta\rho g $ is finite.)

In a fully periodic system, the application of the force on the colloids
will cause a continuous and unboundaed acceleration. To prevent this, and
to balance the momentum budget,
an equal an opposite body force density is applied uniformly to fluid
sites at each time step. In a system with a ``bottom'' (for example, a
flat wall perpendicular to the gravitational force) no such correction
is required.


\subsubsection{Magnetic field}

The the presence of a uniform external magnetic field $\mathbf{B}$,
a particle with dipole $\mathbf{s}$ experiences a torque
$\mathbf{s} \times \mathbf{B}$.

\subsubsection{Electric Field}

In the presence of a uniform external electric field $\mathbf{E}$, a
particle with charge $q$ experiences a body force $q\mathbf{E}$.



% TODO
% Yukawa
% Bonds, angles, etc
% Squirmer
% Sub grid dynamics
% solid-solid links

% End section
\vfill
\pagebreak
