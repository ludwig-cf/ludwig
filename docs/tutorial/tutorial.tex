%%%%%%%%%%%%%%%%%%%%%%%%%%%%%%%%%%%%%%%%%%%%%%%%%%%%%%%%%%%%%%%%%%%%%%%%%%%%
%
%  ludwigTutorial.tex
%
%  Introductory tutorial for new users of Ludwig. 
%
%  Edinburgh Soft Matter and Statistical Physics Group and
%  Edinburgh Parallel Computing Centre and
%  Department of Physics, University of Strathclyde
%
%  Contributing authors:
%  Oliver Henrich (oliver.henrich@strath.ac.uk)
%  Fraser Mackay (s1026487@sms.ed.ac.uk)
%
%  (c) 2008-2017 The University of Edinburgh
%
%%%%%%%%%%%%%%%%%%%%%%%%%%%%%%%%%%%%%%%%%%%%%%%%%%%%%%%%%%%%%%%%%%%%%%%%%%%%

\documentclass[11pt,twoside,a4paper]{article}

\usepackage{amsmath}
\usepackage{moreverb}
\usepackage{lscape}
\usepackage{epic}
\usepackage[pdftex]{graphicx}
\usepackage{bm}
\usepackage{hyperref}
\usepackage{color}
\usepackage{listings}
\usepackage{mathtools}
\usepackage{enumerate}

\usepackage{geometry}
 \geometry{
 a4paper,
 left=3.1cm,
 right=3.1cm,
 top=2.25cm,
 bottom=2.25cm 
}

\setlength{\parindent}{0pt}
\setlength{\parskip}{\smallskipamount}

\newcommand{\inputkey}[1]{\framebox{\textbf{\texttt{#1}}}}
\newcommand{\e}[1]{\cdot10^{#1}}
\newcommand{\beq}{\begin{equation}}
\newcommand{\eeq}{\end{equation}}
\newcommand{\beqa}{\begin{eqnarray}}
\newcommand{\eeqa}{\end{eqnarray}}
\newcommand{\com}[1]{\textcolor{red}{#1}}
\newcommand{\plag}[1]{\textcolor{green}{#1}}
\newcommand{\cur}[1]{{\textit{#1}}}

\definecolor{terminalcolour}{gray}{0.96}

\lstdefinestyle{terminalverbatim}{
  basicstyle=\small\ttfamily,
  columns=flexible,
  backgroundcolor=\color{terminalcolour},
  xleftmargin=0pt
}


\begin{document}


\setcounter{page}{1}

% We're not including the table of contents just at the moment
%\tableofcontents

\newpage
\lstset{style=terminalverbatim}

\setcounter{page}{1}

% These redefinitions are just compressing the spacing a little.

\makeatletter
\renewcommand*{\section}{%
\@startsection {section}{1}{\z@}%
  {-1.75ex \@plus -0.5ex \@minus -.1ex}%
  {1.15ex \@plus.1ex}%
  {\normalfont\Large\bfseries}%
}
\renewcommand*{\subsection}{%
\@startsection {subsection}{2}{\z@}%
  {-1.75ex \@plus -0.5ex \@minus -.1ex}%
  {1.15ex \@plus.1ex}%
  {\normalfont\large\bfseries}%
}
\renewcommand*{\subsubsection}{%
\@startsection {subsubsection}{1}{\z@}%
  {-1.75ex \@plus -0.5ex \@minus -.1ex}%
  {1.15ex \@plus.1ex}%
  {\normalfont\normalsize\bfseries}%
}

% Table of Contents
\pagenumbering{roman}
\title{Ludwig User Tutorial}
\author{Fraser Mackay and Oliver Henrich}
\maketitle 
\tableofcontents
\clearpage

% Sections
\pagenumbering{arabic}

\section{Introduction}

This tutorial document takes new users of the \texttt{Ludwig} code through the process of how to obtain 
and build the code before outlining some short tutorial exercises where simulations 
of simple example cases will be run and visualised.
It is assumed that the reader is familar with UNIX-based operationg systems and has 
a basic knowledge of hydrodynamics, complex fluids, and to some extent statistical physics. 
This knowledge will be required to make sense of the input and output involved in using the code. 

The system requirements are an up-to-date version of a C compiler as well as working installations 
of Apache's Subversion (SVN) version control system, the visualisation tool
ParaView (\hyperref[ParaView]{https://www.paraview.org}) and the MPI-library 
for compiling the parallel version of the code.

\section{Access and Compilation}

\subsection{Obtaining the Code}
\label{sec:getCode}

The \texttt{Ludwig} project is currently hosted at our central repository 
at CCPForge:\\
(\hyperref[CCPForge]{https://ccpforge.cse.rl.ac.uk}).\\
For full access to the repository please open a CCPForge account and request
to join the project \texttt{Ludwig}. The following command grants anonymous access:

\begin{lstlisting}[style=terminalverbatim]
$ svn checkout https://ccpforge.cse.rl.ac.uk/svn/ludwig
\end{lstlisting}

A copy of the \texttt{Ludwig} repository will now be available in your current directory. 
You should see:

\begin{lstlisting}
$ ls ludwig
branches  tags  trunk
\end{lstlisting}
which is the usual svn layout for trunk, branches and tagged versions.

A full documentation of \texttt{Ludwig} can be obtained by running 
the command \texttt{make pdf} twice in the \texttt{/trunk/docs} directory:

\begin{lstlisting}[style=terminalverbatim]
$ cd /ludwig/trunk/docs/
$ make pdf
\end{lstlisting}

\subsection{Configuration and Build}

The following section outlines the configuration and compilation 
procedure. The configuration step is based on the GNU Compiler Collection gcc.

\begin{enumerate}
\item Go to the directory \texttt{/ludwig/trunk/}: \\
\begin{lstlisting}
$ cd ludwig/trunk
\end{lstlisting}
\item Copy the configuration file \texttt{lunix-gcc-default.mk} to the \texttt{/trunk} directory 
and rename to \texttt{config.mk}: \\
\begin{lstlisting}
$ cp config/lunix-gcc-default.mk ./config.mk
\end{lstlisting}
\item Issue \texttt{make} in \texttt{/targetDP}: \\
\begin{lstlisting}
$ cd targetDP/
$ make 
\end{lstlisting}
This creates dummy functions for calls to the targetDP library.
\item Issue \texttt{make} in \texttt{/mpi\_s}: \\
\begin{lstlisting}
$ cd ../mpi_s/
$ make 
\end{lstlisting}
This creates dummy functions for calls to the MPI-library,
which is required for serial compilation.
\item Edit \texttt{Makefile} in \texttt{/src}. To switch off 
the assertions set \texttt{OPTS = -DNP\_D3Q6 -DNDEBUG}. This is
important for production runs (see also comment in \texttt{Makefile}): \\
\begin{lstlisting}
$ cd ../src/
$ vim Makefile
...

include ../Makefile.mk

MAIN = main
EXECUTABLE = Ludwig.exe
LIBRARY = libludwig.a

OPTS = -DNP_D3Q6 -DNDEBUG
LIBS = -L../targetDP -ltarget -lm
INC = -I. -I ../targetDP
...
\end{lstlisting}
\item To compile the code in serial issue: \\
\begin{lstlisting}
$ make serial
\end{lstlisting} 
To compile the parallel version of the code issue: \\
\begin{lstlisting}
$ make mpi
\end{lstlisting} 
This creates the executable file \texttt{Ludwig.exe} in \texttt{/src}.
\end{enumerate}



%\clearpage
%\vfill\pagebreak
%%%%%%%%%%%%%%%%%%%%%%%%%%%%%%%%%%%%%%%%%%%%%%%%%%%%%%%%%%%%%%%%%%%%%%%%%%%%
%
%  references.tex
%
%  Bibliography
%
%  $Id: references.tex 1902 2013-01-31 15:11:10Z ohenrich $
%
%  Edinburgh Soft Matter and Statistical Physics Group and
%  Edinburgh Parallel Computing Centre
%
%  (c) 2011-2016 The University of Edinburgh
%  Kevin Stratford (kevin@epcc.ed.ac.uk)
%
%%%%%%%%%%%%%%%%%%%%%%%%%%%%%%%%%%%%%%%%%%%%%%%%%%%%%%%%%%%%%%%%%%%%%%%%%%%

\vfill
\pagebreak

\addcontentsline{toc}{section}{References}

\bibliographystyle{plain}
\begin{thebibliography}{99}


\bibitem{adhikari-desplat2005}
Adhikari, R. J.-C. Desplat, and K. Stratford,
Sliding periodic boundary conditions for lattice Boltzmann and lattice
kinetic equations,
\texttt{arXiv:cond-mat/0503175v1} (2005).

\bibitem{apacheAPR} 
Apache Portable Runtime Project ``APR's Version Numbering''.
See, e.g., \texttt{https://apr.apache.org/versioning.html)}
(accessed November 2015).

\bibitem{adhikari2005}
Adhikari, R., K. Stratford. M.E. Cates, and A.J. Wagner,
Fluctuating Lattice Boltzmann,
\textit{Europhys. Lett.}, \textbf{71}, 473 2005.

\bibitem{ald98}
Aidun, C.K., Y. Lu, and E.-J. Ding,
Direct analysis of particulate suspensions with inertia using the
discrete Boltzmann equation,
\textit{J. Fluid Mech.}, \textbf{373}, 287, 1998.

\bibitem{allen-tildesley}
M.P. Allen and D.J. Tildesley,
\textit{Computer simulation of liquids},
Oxford University Press (1987).

\bibitem{batchelor}
G.K. Batchelor
\textit{An Introduction to Fluid Mechanics},
Cambridge University Press (1967).

\bibitem{beris-edwards}
Beris, A.N., and B.J. Edwards,
\textit{Thermodynamics of flowing systems with internal microstructure},
Oxford University Press (1994).

\bibitem{blake}
J.R. Blake, A spherical envelope approach to ciliary propusion,
\textit{J. Fluid Mech.}, \textbf{46}, 199, 1971.

\bibitem{cahn-hilliard-1958}
J.W. Cahn and J.E. Hilliard,
Free energy of a nonuniform system. I. Interfacial free energy,
\textit{J. Chem. Phys.} \textbf{28} 258--267 (1958).

\bibitem{cates_scaling}
M. E. Cates, J.-C. Desplat, P. Stansell, A.J. Wagner, K. Stratford,
R. Adhikari, and I. Pagonabarraga,
Physical and Computational Scaling Issues in Lattice Boltzmann
Simulations of Binary Fluid Mixtures,
\textit{Phil. Trans. Roy. Soc. A}, \textbf{363}, 1917 (2005). 

\bibitem{chaikin-lubensky}
P.M. Chaikin and T.C. Lubensky,
\textit{Principles of condensed matter physics}, Cambridge University
Press (1995).

\bibitem{chunladd}
B. Chun and A.J.C. Ladd,
Interpolated boundary condition for lattice Boltzmann simulations in
narrow gaps,
\textit{Phys. Rev. E}, \textbf{75}, 066705, 2007.

\bibitem{cj98}
Cichocki, B., and R.B. Jones,
Image representation of a spherical particle near a hard wall,
\textit{Physica A}, \textbf{258}, 273, 1998.

\bibitem{chang}
C.-H. Chang and E.I. Franses,
Adsorption dynamics of surfactants at the air/water interface:
a critical review of mathematical models, data, and mechanisms,
\textit{Colloids and Surfaces A}, \textbf{100} 1, 1995.

\bibitem{degennes-prost2002}
de Gennes, J.-G., and J. Prost,
\textit{The physics of liquid crystals},
Oxford University Press (2002).

\bibitem{desplat2001}
Desplat, J.-C., I. Pagonabarraga, and P. Bladon,
LUDWIG: A parallel lattice-Boltzmann code for complex fluids.
\textit{Comput. Phys. Comms.}, \textbf{134}, 273, 2001.

\bibitem{diamant}
H. Diamant and D. Andelman,
Kinetics of surfactant adsorption at fluid/fluid
interfaces: non-ionic surfactants,
\textit{Europhys. Lett.} \textbf{34}, 575 (1996).

\bibitem{diamant96}
H. Diamant and D. Andelman,
Kinetics of surfactant adsorption at fluid-fluid interfaces,
\textit{J. Phys. Chem.}, \textbf{100} 13732, 1996.

\bibitem{fournier2005}
J.-B. Fournier and P. Galatola,
Modeling planar degenerate wetting and anchoring in nematic liquid
crystals,
\textit{Europhys. Lett.}, \textbf{72} 403--409 (2005).

\bibitem{eastoe}
J. Eastoe and J.S. Dalton,
Dynamic surface tension and adsorption mechanims of surfactants
at the air-water interface,
\textit{Advances in Colloid and Interface Science}, \textbf{85}
13, 2000.

\bibitem{ginzburg}
I. Ginzburg and D. d'Humi\`eres,
Multireflection boundary conditions for lattice Boltzmann models,
\textit{Phys. Rev. E}, \textbf{68}, 066614, 2003.

\bibitem{gray2013}
A. Gray and K. Stratford,
TargetDP reference.

\bibitem{h59}
Hasimoto, H., On the periodic fundamental solutions of the Stokes
equation and their application to viscous flow past a cubic array
of spheres.
\textit{J. Fluid Mech.}, \textbf{5}, 317.

\bibitem{heemels}
Heemels, M.W., M.H.J. Hagen, and C.P. Lowe, Simulating solid colloidal
particles using the lattice-Boltzmann method,
\textit{J. Comp. Phys.}, \textbf{164}, 48, 2000.

\bibitem{ieee-208}
IEEE Standard 208-2005, IEEE Standard for Software Configuration Management
Plans.
See \texttt{http://ieeexplore.ieee.org/xpl/standards.jsp}
(accessed November 2015).

\bibitem{ieee-828-2012}
IEEE Standard 828-2012. IEEE Standard for Configuration
Management in Systems and Software Engineering. 
See \texttt{http://ieeexplore.ieee.org/xpl/standards.jsp}
(accessed November 2015).

\bibitem{jo84}
Jeffrey, D.J., and Y. Onishi,
Calculation of the resistance and mobility functions for the two
unequal rigid spheres in low-Reynolds-number flow,
\textit{J. Fluid Mech.}, \textbf{139}, 261, 1984.

\bibitem{kendon2001}
Kendon, V.M., M.E. Cates, I. Pagonabarraga, J.-C. Desplat, and
P. Bladon,
Inertial effects in three dimensional spinodal decomposition of
a symmetric binary fluid mixture: A lattice Boltzmann study,
\textit{J. Fluid Mech.}, \textbf{440}, 147 (2001).

\bibitem{l94a}
Ladd, A.J.C., Numerical simulations of particulate suspensions
via a discretised Boltzmann equation. Part 1. Theoretical foundation,
\textit{J. Fluid. Mech.}, \textbf{271}, 285, 1994.

\bibitem{l94b}
Ladd, A.J.C., Numerical simulations of particulate suspensions
via a discretised Boltzmann equation. Part 2. Numerical results,
\textit{J. Fluid. Mech.}, \textbf{271}, 311, 1994.

\bibitem{l96a}
Ladd, A.J.C., Sedimentation of homogenous suspensions of non-Brownian
spheres,
\textit{Phys. Fluids}, \textbf{9}, 491. 1996.

\bibitem{l96b}
Ladd, A.J.C., Hydrodynamic screening in sedimentating suspensions
of non-Brownian spheres,
\textit{Phys. Rev. Lett.}, \textbf{76}, 1392, 1996.

\bibitem{lv01}
Ladd, A.J.C., and R. Verberg,
Lattice-Boltzmann simulations of particle-fluid suspensions,
\textit{J. Stat. Phys.}, \textbf{104}, 1191, 2001.

\bibitem{lees-edwards1972}
A.W. Lees and S.F. Edwards,
The computer study of transport processes under extreme conditions,
\textit{J. Phys C} \textbf{5} 1921--1929 (1972).

\bibitem{lipanmiller}
H. Li, C. Pan, and C.T. Miller,
Pore-scale investigation of viscous coupling effects for two-phase
flow in porous media,
\textit{Phys. Rev. E}, \textbf{72}, 026705, 2005.

\bibitem{lighthill}
M.J. Lighthill,
On the squirming motion of nearly spherical deformable bodies through
liquid at very small Reynolds numbers,
\textit{Comm. Pure Appl. Math.}, \textbf{5}, 109, 1952.

\bibitem{isaac}
I. Llopis Fust\'e,
\textit{Hydrodynamic cooperativity in micro-swimmer suspensions},
Ph.D. Thesis, University of Barcelona, 2008.

\bibitem{mpi-standard}
Message Passing Interface Forum. MPI: A Message Passing Interface Standard
Version 1.3 (2008).

\bibitem{nguyen-ladd2002}
Nguyen, N.-Q., and A.J.C. Ladd, Lubrication corrections for
lattice-Boltzmann simulations of particle suspensions,
\textit{Phys. Rev. E}, \textbf{66}, 046708, 2002.

\bibitem{papanastasiou}
T. paapnastasiou, G. Georgiou, and A. Alexandrou,
\textit{Viscous Fluid Flow},
CRC Press, Boca Raton, Florida, 2000.

\bibitem{paraview}
Paraview. See \texttt{http://www.paraview.org/}. Accessed 2011.

\bibitem{rapaport1995}
Rapaport, D.C., \textit{The Art of Molecular Dynamics Simulation},
Cambridge University Press, 1995.


\bibitem{j-stat-phys-2005}
K. Stratford, R. Adhikari, I. Pagonabarraga, and J.-C. Desplat,
Lattice Boltzmann for binary fluids with suspended colloids,
\textit{J. Stat. Phys.} \textbf{121} 163--178 (2005).

\bibitem{succi}
S. Succi, \textit{The lattice Boltzmann equation and beyond},
Oxford University Press, Oxford, 2001.

\bibitem{swift1996}
M.R. Swift, E. Orlandini, W.R. Osborn, and J.M. Yeomans,
Lattice Boltzmann simulation of liquid-gas and binary fluid systems,
\textit{Phys. Rev. E}, \textbf{54} 5041 (1996).

\bibitem{edo1}
M. Venuroli and E.S. Boek,
Two-dimensional lattice-Boltzmann simulations of single phase
flow in a pseudo two-dimensional micromodel,
\textit{Physica A}, \textbf{362}, 23, 2006.

\bibitem{vandergraaf}
R.G.M. van der Sman and S. van der Graaf,
Diffuse interface model of surfactant adsorption onto flat and
droplet interfaces,
\textit{Rheol. Acta} \textbf{46} 3 (2006).

\bibitem{theissengompper}
O. Theissen and G. Gompper,
Lattice Boltzmann study of spontaneous emulsification,
\textit{Eur. Phys. J. B}, \textbf{11} 91 (1999).

\bibitem{wagner-pagonabarraga2002}
A.J. Wagner and I. Pagonabarraga,
Lees-Edwards boundary conditions of lattice Boltzmann,
\textit{J. Stat. Phys.} \textbf{107} 521--537 (2002).

\bibitem{wardtordai}
A.F.H. Ward and L. Tordai,
\textit{J. Chem. Phys.} \textbf{14} 453, 1946.

\bibitem{skarabot}
M. Skarabot, M. Ravnik, S. Zumer, U. Tkalec, I. Poberaj, D. Babic, N. Osterman and I. Musevic,
\textit{Phys. Rev. E} \textbf{76}, 051406 (2007).

\bibitem{wright-mermin}
Wright, D.C> and N.D. Mermin,
\textit{Rev. Mod. Phys.} \textbf{61}, 385--432 (1989).

\bibitem{Lyklema} J. Lyklema {\em Fundamentals of Interface and Colloid Science} Academic Press     (1995).
\bibitem{Mafe} S. Maf\'e, J.A. Manzanares, J. Pellicer, {\textit J. Electroanal. Chem.} {\textbf 241}, 5    7-77 (1988).
\bibitem{Capuani} F. Capuani, I. Pagonabarraga, D. Frenkel, {\textit J. Chem. Phys.} {\textbf 121}, 973-    986 (2004).
\bibitem{Rotenberg} B. Rotenberg, I. Pagonabarraga, D. Frenkel, {\textit Farad. Discuss.} {\textbf 144},     223-243 (2010).
\bibitem{Landau-ED} L.D. Landau, E.M. Lifshitz, {\textit Electrodynamics of Continuous Media}, \S 15    , 2nd ed., Pergamon Press, Oxford, UK (1984).
\bibitem{Melcher} J.R. Melcher, {\textit Continuum Electromechanics}, \S 3.10, MIT Press, Cambridge,     MA, USA (1981).\\
downloadable from:\\
\url{http://ocw.mit.edu/ans7870/resources/melcher/resized/cem_811.pdf}
\bibitem{Landau-EL} L.D. Landau, E.M. Lifshitz, {\textit Theory of Elasticity}, \S 3 \& \S 16, 3rd e    d., Butterworth-Heinemann, Oxford, UK (1986).



\end{thebibliography}




\end{document}
